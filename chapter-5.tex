\chapter{可见光多波段自适应通信系统硬件设计}
\section{引言}
我们在前面四章介绍了可见光通信的基本原理及关键技术,特别针对自适应传输这个核心重点研究了信道估计及比特功率分配算法,并且针对本课题对应的硬件平台的实际情况进行了别要的仿真,选出了合适的技术方案,如使用低复杂度的LS算法进行信道估计、使用高精度的EVM方法进行信噪比估计、使用专为可见光通信设计的Improved-SBLA比特功率分配算法得到自适应参数。本章将对可见光通信的硬件系统做一个简要的介绍,还将概述自适应模块的逻辑设计。
\section{硬件系统概述}
本课题对应的硬件演示平台如图1所示,该系统目前已经实现了“编译级”的自适应传输,所谓“编译级”就是代码支持通过改变调制参数然后需要再编译来实现调制的改变,而真正的自适应传输系统因为时间紧迫及反向链路方案尚未确定等因素没有完成。不过本系统已有了自适应传输的雏形了,只是信道估计、计算自适应参数、改变调制等需要离线进行,下面对该系统进行概述。

首先信源比特通过以太网接口(UDP协议)按帧发送到用于基带处理的FPGA芯片(整个传输过程都是按帧进行的,并且用于同步和信道估计的ZC序列符号只在帧头处放置,整个帧中所有的OFDM符号都使用这个ZC序列估计出来的信道参数解调),接着在FPGA中完成扰码、信道编码、调制和IFFT等数字处理过程,然后将时域数字信号输入到数字模拟变换器(Digital to Analogue Converter,DAC)变成模拟信号,最后该模拟信号加上偏置电流之后去驱动LED灯,整个发射过程完成。接收端通过PD接收LED光信号,并将光信号强弱的变化转换成电信号的大小,然后将此模拟电信号送入模拟/数字变换器(Analog to Digital Converter,ADC)中抽样量化为数字信号,再送到接收端基带处理FPGA进行解调、解码和校验等操作,最后输出接收到的帧到接收端计算机。
\section{自适应模块FPGA方案设计}
\section{本章小结}